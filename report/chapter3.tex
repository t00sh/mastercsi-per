\chapter{Fuzzing de binaires quelconques}

Jusqu'à présent, il nous a été nécessaire d'avoir accès au code source d'un programme cible pour pouvoir le tester avec AFL.
En effet, l'instrumentation du binaire prend place lors de la compilation effectuée par \lstinline{afl-gcc}, à partir des sources de l'application victime.
Cela peut être problématique lorsque l'on souhaite fuzzer des logiciels dont le code source n'est pas accessible (dans le cas de logiciels propriétaires par exemple).
Pour répondre à ce problème, AFL peut s'appuyer sur le mode "user space emulation" de QEMU\footnote{\url{https://www.qemu.org/}}, qu'il utilise pour effectuer l'instrumentation du binaire "à la volée" (pendant l'exécution d'\lstinline{afl-fuzz}).

De plus, nous avons toujours implicitement considéré des binaires comme étant compilés pour s'exécuter sur des architectures x86.
Encore, le mode QEMU proposé par AFL peut nous aider à élargir notre champ d'action, en se servant de QEMU pour supporter d'autres plateformes (PowerPC, ARM, SPARC...\footnote{\url{https://qemu.weilnetz.de/doc/qemu-doc.html\#QEMU-System-emulator-for-non-PC-targets}.}).
Une autre solution possible serait d'utiliser une version d'AFL différente que celle que nous avons présenté jusqu'à présent.
Cette alternative injecterait du code assembleur relatif à la plateforme ciblée (en lieu et place de l'assembleur x86).
C'est par exemple la méthode employée par le projet \lstinline{android-afl}\footnote{\url{https://github.com/ele7enxxh/android-afl}}, cherchant à fuzzer des applications mobiles Android.

\section{QEMU mode}

Comme nous venons brièvement de l'annoncer, QEMU peut être utilisé avec AFL pour :
\begin{itemize}
  \item{} Fuzzer des binaires sans avoir accès à leur code source ;
  \item{} Fuzzer des binaires qui ont été compilés pour s'exécuter sur des plateformes autres que x86.
\end{itemize}

\begin{quote}
QEMU est un logiciel libre de machine virtuelle, pouvant émuler un processeur et, plus généralement, une architecture différente si besoin. Il permet d'exécuter un ou plusieurs systèmes d'exploitation via les hyperviseurs KVM et Xen, ou seulement des binaires, dans l'environnement d'un système d'exploitation déjà installé sur la machine.\footnote{\url{https://fr.wikipedia.org/wiki/Qemu}}
\end{quote}

\subsection{"user space emulation"}

Parmi les différentes fonctionnalités que QEMU propose, on s'intéresse ici à l'émulation en espace utilisateur ("user space emulation")\footnote{\url{https://qemu.weilnetz.de/doc/qemu-doc.html\#QEMU-User-space-emulator}} \cite{qemu}.
Par cela, il devient possible d'exécuter des binaires Linux ou BSD quelconques sur n'importe quelle architecture permettant d'exécuter QEMU, en l'occurrence, une architecture x86 de nos ordinateurs personnels ou de l'environnement du CREMI.

Dans ce mode, QEMU émule uniquement le CPU, sur lequel il exécute le binaire ciblant une autre architecture que celle sur laquelle QEMU tourne.
Il ne simule pas le matériel, ni le noyau.
Pour fonctionner, il intercepte les différents appels systèmes qui surviennent et les transmet au noyau du système hôte.
\footnote{\url{https://ownyourbits.com/2018/06/13/transparently-running-binaries-from-any-architecture-in-linux-with-qemu-and-binfmt_misc/}}

\paragraph{}
Sans établir une description complète du fonctionnement de QEMU, quelques mots sont cependant nécessaires pour comprendre par la suite la manière dont AFL effectue son instrumentation, essentielle au suivi du flot d'exécution du programme.

QEMU est capable de découper un binaire en différents ``basic blocs'', chaque fin de bloc étant repérée grâce à une instruction modifiant le compteur de programme (par exemple les instructions \lstinline{jmp} ou encore \lstinline{ret} du jeu d'instructions x86).
Il utilise alors un générateur de code : le \lstinline{TCG} (Tiny Code Generator\footnote{\url{https://wiki.qemu.org/Documentation/TCG}}) pour traduire les basic blocs du programme utilisant des instructions de l'architecture cible vers des ``Translated Blocs'' utilisant des instructions de l'architecture hôte.

\subsection{Instrumentation par AFL}

AFL utilise une version patchée de QEMU permettant d'implémenter les opérations dont il a besoin.
En particulier : arriver à suivre le flot d'exécution du programme grâce à une forme d'instrumentation, et gérer le modèle du ``fork server''\footnote{Se référer à la section \ref{fork-server}.}.
Les différents changements apportés à QEMU sont disponibles dans le dossier \path{qemu_mode/patches/} d'AFL\footnote{\url{https://github.com/mirrorer/afl/tree/master/qemu_mode/patches}}.

Décrite dans le whitepaper d'AFL \cite[section 12 : ``Binary only instrumentation'']{technical-details}, voici à quoi ressemble l'instrumentation mise en place sur les ``Translated Blocs'' générés par QEMU :
\begin{lstlisting}[language=C]
  if (block_address > elf_text_start && block_address < elf_text_end) {
    cur_location = (block_address >> 4) ^ (block_address << 8);
    shared_mem[cur_location ^ prev_location]++;
    prev_location = cur_location >> 1;
  }
\end{lstlisting}

La différence avec l'instrumentation classique réalisée avec \lstinline{afl-gcc} est que la \lstinline{cur_location} n'est plus une valeur aléatoire, mais calculée en fonction de l'adresse du basic bloc.
Il serait en effet compliqué de faire pareil sans patcher plus profondément QEMU car il faudrait générer une valeur unique pour chaque basic bloc.
Or ici, le code inséré dans QEMU est appelé à chaque exécution du ``Translated Bloc'' et on perdrait le caractère unique de cette valeur (aussi, le cache de QEMU ne conserve pas forcément l'ensemble de ces ``Translated Bloc'' en mémoire. On peut donc difficilement stocker une valeur aléatoire dans cette structure).

Ainsi, pour avoir une valeur unique pour chaque basic bloc, AFL utilise l'adresse de celui-ci pour l'identifier.
Ces adresses étant généralement alignées pour optimiser le chargement des instructions dans le CPU et pour des raisons d'architectures : elles contiennent un nombre important de bits à zéro.
Pour masquer cette alignement, AFL utilise des shifts pour identifier un basic bloc.


\paragraph{}
La première traduction d'un basic bloc est toujours la plus lente.
Chaque traduction est placée dans un cache, permettant à QEMU de retrouver rapidement des ``Translated Blocs'' s'ils ont déjà été générés au cours de l'exécution.
À travers le mode QEMU, AFL utilise un ``fork server'' différent de celui présenté dans la section \ref{fork-server} : ici, est créé un canal de communication entre l'émulateur et le processus parent, permettant de générer un cache de traduction partagé avec les différents processus fils.
Ainsi, lorsqu'un nouveau bloc est atteint, le parent est averti, puis le cache est mis à jour, et chaque fils peut alors bénéficier de la traduction qui vient d'être effectuée.
Ainsi, le mode QEMU est deux à cinq fois plus lent que l'utilisation normale d'AFL.

\section{AFL pour d'autres architectures}

AFL a été pensé à la base pour cibler les architectures x86 et x86-64 uniquement.
L'instrumentation ajoutée par AFL étant fortement dépendante du processeur cible (le code de \lstinline{afl-as} a de nombreuses parties écrites directement en assembleur), il est nécessaire de réécrire de nombreux éléments de AFL pour en supporter un nouveau type.

C'est ce qu'a fait par exemple android-afl\footnote{\url{https://github.com/ele7enxxh/android-afl}}, ciblant l'architecture ARM et en particulier le système d'exploitation mobile Android.
Ce projet a complètement réécrit des parties du code de \lstinline{afl-as} afin de rajouter de l'instrumentation en assembleur ARM que ça soit pour les parties instrumentant les changements de ``basic blocs'' ou la communication avec le ``fork server''.

Un autre projet, afl-other-arch\footnote{\url{https://github.com/shellphish/afl-other-arch}}, utilise le mode QEMU utilisateur afin d'émuler l'architecture cible.
Cela permet de supporter alors toutes les architectures disponibles sur QEMU, qui sont très nombreuses.

Ainsi, QEMU peut être à la fois utilisé pour fuzzer un logiciel dont on n'a pas le code source et pour émuler une architecture différente de x86 et x86-64.
