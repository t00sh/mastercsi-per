\chapter*{Conclusion}
\addcontentsline{toc}{chapter}{\protect\numberline{}Conclusion}

Notre étude explique le fonctionnement d'AFL, d'ASan, et nous présentons une
application concrète sur une suite de logiciels modernes : Radare2.

La rapidité et la facilité avec laquelle AFL expose des problèmes confirme
l'efficacité du fuzzing comme outil pour trouver des bugs dans un programme.
Comme nous l'avons constaté, l'utilisation de ASan apporte des résultats
supplémentaires et se combine très bien avec l'utilisation d'un fuzzer.

Cependant, nous avons aussi vu les limites d'une telle approche.
En effet, sans accès au code source du programme que l'on souhaite tester, l'instrumentation à la compilation est impossible.
De plus, de nos jours, une quantité non négligeable de programmes ne ciblent plus des architectures x86, notamment les applications mobiles.
Il est alors nécessaire de contourner ces problèmes, en utilisant le mode QEMU proposé par AFL, dont les performances sont alors dégradées comparées à celles obtenues par son fonctionnement "classique".
Il est aussi possible d'adapter la méthode suivie par AFL pour cibler d'autres plateformes, comme par exemple ARM, en compilant sur cette architecture et injectant le code instrumenté écrit avec de l'assembleur adéquat ; mais cette technique reste marginale, et peu de projets semblent l'exploiter.
Mentionnons aussi la problématique des programmes étant exécutés sur des systèmes Windows, pour lesquels existe une version d'AFL, originaire du Project Zero : WinAFL\footnote{\url{https://github.com/googleprojectzero/winafl}}.

Plus généralement, AFL n'est pas le seul outil utilisé dans le cadre du fuzzing.

Le moteur de fuzzing libFuzzer\footnote{\url{https://llvm.org/docs/LibFuzzer.html}}
est également énormément utilisé dans ce domaine.
Il s'agit d'une bibliothèque basée sur la chaîne de compilation LLVM, et en
particulier utilise l'instrumentation apportée par son outil SanitizerCoverage,
pour son exploration du code à tester.
Ce fuzzer nécessite de créer un point d'entrée dans le programme à tester, une
"fonction cible", à laquelle libFuzzer va alors s'attaquer pour chercher des bugs.
On peut observer un gain de performance notable par rapport à AFL.

Un autre projet développé par Google à l'origine pour son navigateur Chrome,
a également vu le jour : ClusterFuzz.
C'est une infrastructure de fuzzing destinée à accompagner la recherche de
bugs en continue lors des phases de développement d'un programme.
Lancé en 2016, OSS-Fuzz\footnote{\url{https://github.com/google/oss-fuzz}}
est une plateforme utilisant ClusterFuzz comme back end, mise en place pour
aider les projets open source à trouver des bugs. Le code source de
ClusterFuzz\footnote{\url{https://github.com/google/clusterfuzz}}
a d'ailleurs été lui-même ouvert début février 2019.
Cette infrastructure utilise à la fois AFL et libFuzzer pour le fuzzing,
et permet de lancer des instances sur un nombre important de CPU (le projet
utilise en tout plus de 25000 cœurs).
Un retour d'expérience sur le fuzzing de ImageMagick et GraphicsMagick grâce à OSS-Fuzz
peut être consulté sur le blog de Alex Gaynor
\footnote{\url{https://alexgaynor.net/2019/feb/05/notes-fuzzing-imagemagick-graphicsmagick/}}.

Frustré par le nombre de bugs présents dans les projets open source, Hanno
Böck a de son côté aussi souhaité aider la communauté en lançant
le Fuzzing Project\footnote{\url{https://fuzzing-project.org}}. L'idée étant
d'appliquer lui même une série de tests sur une partie des projets open
source les plus utilisés et de remonter les bugs aux contributeurs concernés.
Grâce à ce projet, Hanno Böck a déjà remonté des problèmes à de nombreux
logiciels tels que Apache\footnote{CVE-2017-9798}, MatrixSSL\footnote{CVE-2016-6885},
ou libarchive\footnote{CVE-2015-8934}.
