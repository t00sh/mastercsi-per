\chapter{Présentation d'AFL}

Le fuzzing est une technique très utilisée en informatique pour découvrir
des bugs dans un programme. Elle consiste à générer des entrées aléatoires
ou construites intelligemment, dans le but d'essayer de faire crasher le
programme.

American fuzzy lop (AFL) est un logiciel open source de fuzzing qui utilise
un algorithme génétique pour générer des entrées à envoyer au programme à
fuzzer. Le fonctionnement d'un algorithme génétique consiste à utiliser
une population qui va évoluer pour essayer de garder "les meilleurs" de la
population et également créer de nouveaux bons candidats.

Typiquement ici, les individus de la population seront des entrées au
programme à tester. Pour savoir si un individu (une entrée) est intéressant
à garder pour la population, on regardera ici si il produit une
exécution nouvelle du programme par rapport à la population précédente. La
section \ref{nouveaux_comportements} explique comment les individus sont
gardés ou rejetés lors du déroulement de l'algorithme génétique.

Pour créer de nouveaux individus à la population, les algorithmes
génétiques utilisent plusieurs techniques, par exemple ils peuvent
"fusionner" deux individus, ou faire des mutations à partir d'un individu
donné.

\section{Instrumentation}\label{instrumentation}

Toute l'exploration d'un fuzzer comme AFL consiste à  essayer de trouver le
plus de chemins d'exécution différents dans le binaire.

Il faut donc pour cela un moyen de différencier quels chemins ont été pris
entre deux exécutions données.

Lorsqu'un programme est compilé avec AFL, ce dernier instrumente donc le
binaire autour des points de branchement (conditions, boucles).

L'instrumentation est ajoutée à la compilation directement au niveau
assembleur à chaque point de branchement du programme, grâce à l'outil
\listineline{afl_as}.

On peut la résumer par le pseudo code suivant :

\begin{lstlisting}
  cur_location = <COMPILE_TIME_RANDOM>;
  shared_mem[cur_location ^ prev_location]++;
  prev_location = cur_location >> 1;
\end{lstlisting}

Un basic bloc d'un programme est une suite d'instruction séquentielle d'un
programme (sans condition, ni boucle). Ainsi, si on note A un point de
branchement dans le programme, et B le prochain point de branchement en
partant de A, le couple (A, B) représente un basic bloc selon la définition 
vue juste avant.

En utilisant ces notations, le tableau \lstlisting{shared_mem[]} peut être vu
comme un dictionnaire dont les clefs sont des couples (A, B) et les valeurs
associées sont le nombre de fois où ce basic bloc a été pris dans
l'exécution associée.

L'idée derrière le fait d'associer une valeur aléatoire à la compilation
pour un point de branchement est de simplifier l'édition de liens des
projets complexes, et aussi d'uniformiser la répartition des couples
(A, B) dans le tableau.

Le tableau \lstinline{shared_mem[]} fait 64 kB, ce qui est un bon compromis
(établi empiriquement) entre une taille suffisant grande pour éviter qu'il y
ait trop de collisions entre deux valeurs aléatoires attribuées, tout en
étant à la fois assez petite pour rentrer dans le cache L2 du processeur 
et ainsi permettre des calculs rapides.

L'opération de shift peut paraître étrange à première vue, mais c'est en
fait une astuce pour pouvoir différencier le basic bloc (A, B) de
son opposé (B, A), et aussi pour différencier les boucles
(A\textasciicircum A serait égal à B\textasciicircum B sinon).

À la fin d'une exécution du programme, l'état du tableau
\lstinline{shared_mem[]} représente donc quels basic blocs ont été accédés
et combien de fois.

\section{Détection des nouveaux comportements}\label{nouveaux_comportements}

Comme dit plus haut, le but d'AFL est d'explorer le plus de chemins
d'exécutions différents. Un moyen simple de savoir si deux exécutions
ont produit deux chemins différents serait de comparer leur trace à
l'aide du tableau décrit dans la section précédente.

Ce n'est cependant pas exactement la méthode retenue par AFL. La raison
étant d'éviter la comparaison parfois couteuse entre des traces
d'exécutions complexes, et également éviter d'avoir une explosion
combinatoire des traces.

AFL utilise donc deux méthodes pour savoir si une nouvelle trace d'exécution
est à garder pour le prochain round d'input ou pas.

\subsubsection{Nouveaux tuples}
Le premier test est de savoir si une nouvelle trace d'exécution contient de
nouveaux tuples (A, B). Par exemple, si on dispose déjà dans notre base
d'entrées de la trace d'exécution :

\#1: A -> B -> B -> A -> C -> D

Et que la trace suivante est produite par une exécution :

\#2: A -> B -> A -> B -> A -> C

Alors cette dernière trace \#2 ne sera pas gardée pour le prochain round
d'input, car elle ne contient pas de nouveaux tuples.

En revanche si on considère la trace :

\#3: A -> B -> C -> D

Celle-ci sera gardée pour le prochain round car elle contient le tuple
(B, C) qui n'apparait pas dans la trace \#1.

\subsubsection{Catégories}

La deuxième façon qu'utilise AFL pour savoir si une trace est
intéressante à conserver consiste à compter le nombre d'apparition de
chaque tuple dans une trace.

AFL range ces nombres dans des catégories grossières :

1, 2, 3, 4-7, 8-15, 16-31, 32-127, 128+

Un changement du nombre d'apparition à l'intérieur d'une catégorie
est rejeté, alors qu'un changement entre deux catégories est jugé
pertinent.

Si on reprend l'exemple \#1 précédent et que l'on considère la trace :

\#4: A -> B -> A -> B

Une première constatation est de voir que la trace \#4 ne contient pas de
nouveaux tuples par rapport à la trace \#1, et on devrait donc s'attendre
à voir cette nouvelle trace rejetée. Cependant si on compte le nombre
d'apparition du tuple (A, B) dans ces deux traces, on remarque que le
nombre passe de 1 à 2, ce qui correspondant à un changement de catégorie
selon la répartition donnée plus haut.

La trace \#4 sera donc conservée pour la prochain round d'input d'AFL.

\section{Évolution de la liste des entrées et réduction du corpus}

Comme expliqué plus haut, chaque entrée envoyée au programme à tester
correspond à un individu de la population de l'algorithme génétique.
Dans de tels algorithmes, cette population garde généralement une taille
de population plus ou moins constante au fil des itérations.

La section \ref{nouveaux_comportements} explique comment sont choisies les
nouvelles entrées à rajouter à la liste. Ce qu'il faut savoir, c'est que
ces nouvelles entrées ne remplacent pas celles de l'itération précédente,
mais les supplémentent simplement.

Cela implique une potentielle redondance dans la liste des entrées : 
Il arrive qu'une nouvelle liste d'entrées ait une couverture des points
de branchement identique à la précédente, mais avec une taille de liste
plus grande.

Cela n'est pas souhaitable, on préférerait avoir une liste d'entrées la
plus petite possible pour accélérer le fuzzing, tout en gardant la même
couverture des différents chemins d'exécutions.

AFL utilise donc un petit algorithme pour réduire la taille de la liste,
tout en gardant la même couverture. Pour cela, il associe à chaque entrée
un score relatif à sa taille et à son temps d'exécution associé. Il prend
ensuite chaque couple (A, B) couvert par la liste d'entrées, choisi l'entrée
ayant le plus petit score pour la prochain liste, et itère pour les couples
restants.

Le corpus se retrouve ainsi réduit en taille de 5 à 10 fois par rapport au
précédent.


\section{Améliorations, optimisations}

Une manière directe d'implémenter un logiciel de Fuzzing est de faire exécuter de multiples instances du programme à tester sur des entrées différentes à l'aide d'\lstinline{execve()}\footnote{\url{https://linux.die.net/man/2/execve}}.
Dès lors, le Fuzzer peut détecter des corruptions mémoires en utilisant \lstinline{waitpid()} \footnote{\url{https://manpage.me/?q=waitpid}} sur le processus fils (le programme à tester ayant reçu une entrée particulière) pour vérifier s'il meurt en recevant un signal \lstinline{SIGSEGV}, \lstinline{SIGABRT} ou autre, de la part du noyau.

Cette approche peut se révéler lente, ou tout du moins inefficace : en effet, chaque exécution du programme à tester passe un temps non négligeable à charger le programme en mémoire, linker les librairies dynamiquement ou attendre le noyau lors de l'appel à \lstinline{execve()}.

AFL fonctionne différemment, et tire avantage de deux optimisations, décrites dans le Whitepaper présentant son fonctionnement (\cite[section 10 : The fork server]{technical-details}): le modèle "fork server" et le mode "persistent".


\subsection{Le modèle "fork server"}\label{fork-server}

Ce modèle est présenté dans l'article "Fuzzing random programs without execve()"\cite{fuzzing-binaries-without-execve} présent sur le blog d'AFL.

\subsubsection{Copy-On-Write (COW)}

COW est une stratégie d'optimisation de la mémoire, qui permet aussi de gagner du temps lors de la création d'un processus par un autre lors d'un \lstinline{fork()}\footnote{\url{https://linux.die.net/man/2/fork}}.

En temps normal, lorsque le processus enfant est créé, sa mémoire est initialisée avec une copie de la mémoire du processus père.
Avec COW, la mémoire du processus enfant pointe vers celle du processus père, mais il ne peut pas y écrire : il n'y a pas besoin de faire de copie, et la mémoire du père reste consistante de son point de vue.
Si l'enfant a besoin d'écrire dans sa mémoire, alors la copie est vraiment faite, et le pointeur est modifié.
COW est un mécanisme transparent pour chacun des processus.

\subsubsection{"fork server"}
Une première technique permettant à AFL de gagner du temps au lancement du programme testé est de "démarrer" celui-ci une seule et unique fois.
Ainsi, le système n'exécute qu'un seul appel à \lstinline{execve()} pour charger le programme en mémoire.

Cependant, pour pouvoir tester le programme sur de multiples instances, AFL a besoin d'un mécanisme pour relancer le programme sur des entrées différentes.
La solution implémentée exécute un appel à \lstinline{fork()} pour créer un processus fils, copie exacte du père contenant le programme à tester, dont l'entrée sera cependant différente.
Cet appel à \lstinline{fork()} permet notamment de tirer avantage du COW présenté précédemment.

Cette implémentation soulève quelques questions.

\paragraph{Quel code exécute l'appel à \lstinline{fork()} ? Où se passe cet appel ?}

L'essentiel se passe au cours de l'exécution du programme cible, par du code injecté lors de l'instrumentation, par \lstinline{afl-gcc}.
% TODO: référence à une section précédente, où est détaillée le fonctionnement de AFL
Comme l'intérêt est d'avoir chaque processus fils s'exécutant sur une entrée, il est nécessaire de mettre en pause le père avant que cette dernière soit traitée.
Généralement, le code sera injecté (et donc exécuté) avant le \lstinline{main()} du programme.

Quand ce point est atteint lors de l'exécution du programme, le code injecté attend des commandes de la part d'\lstinline{afl-fuzz}.
À la réception du message "go", le programme exécute alors son appel à \lstinline{fork()}.
Alors, dans le processus fils, c'est le code originel du programme testé qui reprend la main et poursuit son exécution légitime sur l'entrée qui lui est fournie.

% TODO:
% \paragraph{Comment chaque entrée est passée par AFL au processus fils ?}

\paragraph{Comment \lstinline{afl-fuzz} détecte-t-il les crashs ?}

Avec cette méthode, \lstinline{afl-fuzz} n'est plus le parent direct des processus s'exécutant sur les entrées qu'il fournit.
Il ne peut donc pas utiliser \lstinline{waitpid()} directement pour surveiller les crashs potentiels.

C'est le code injecté dans le programme cible lors de l'instrumentation (celui sur lequel l'exécution est "pausée" au moment du \lstinline{main()}) qui se charge d'attendre avec \lstinline{waitpid()}, et de transférer le résultat de l'exécution du fils, de celui-ci vers \lstinline{afl-fuzz}.

\paragraph{Implémentation}

Le code relatif à cette fonctionnalité est dispersé à travers plusieurs fichiers.
On notera en particulier :
\begin{itemize}
  \item{} Gestion du "fork server" dans \lstinline{afl-fuzz} : \url{https://github.com/mirrorer/afl/blob/master/afl-fuzz.c#L1968-L2255};
  \item{} Exécution du programme par \lstinline{afl-fuzz} \url{https://github.com/mirrorer/afl/blob/master/afl-fuzz.c#L2361-L2382};
  \item{} Et dans \url{https://github.com/mirrorer/afl/blob/master/afl-as.h}, avec les fonctions \lstinline{__afl_forkserver}, \lstinline{__afl_fork_resume}, \lstinline{__afl_fork_wait_loop} et la variable \lstinline{__afl_fork_pid}, le code injecté dans le programme cible.
\end{itemize}

