\documentclass{bredelebeamer}

\usepackage{listings}
%% Lstlisting config
\definecolor{mygreen}{rgb}{0,0.6,0}
\definecolor{mygray}{rgb}{0.86,0.86,0.86}
\definecolor{mymauve}{rgb}{0.58,0,0.82}

\lstset{
  backgroundcolor=\color{white},
  basicstyle=\footnotesize,
  breakatwhitespace=false,
  breaklines=true,
  captionpos=b,
  commentstyle=\color{mygreen},
  deletekeywords={...},
  extendedchars=true,
  frame=single,
  keepspaces=true,
  keywordstyle=\color{blue},
  language={},
  morekeywords={*,...},
  numbers=left,
  numbersep=5pt,
  numberstyle=\tiny\color{mygray},
  rulecolor=\color{black},
  showspaces=false,
  showstringspaces=false,
  showtabs=false,
  stepnumber=0,
  stringstyle=\color{mymauve},
  tabsize=2,
  title=\lstname
}
\usepackage{realboxes}
\usepackage{ulem}
\usepackage{xpatch}

%% lstinline style
\definecolor{mygray}{rgb}{0.86,0.86,0.86}
\makeatletter
\xpretocmd\lstinline{\Colorbox{mygray}\bgroup\appto\lst@DeInit{\egroup}}{}{}
\makeatother


\title[]{Techniques de fuzzing}
\subtitle{Avec American Fuzzy Lop}

\author[Xavier M. - Brendan G. - Simon D.]{Xavier Maso \\ Brendan Guevel \\ Simon Duret}


\institute[]{
  \includegraphics[scale=0.2]{../medias/universite-bordeaux.pdf}
}

\date{21 février 2019}

\begin{document}

\begin{frame}
  \titlepage
\end{frame}

\begin{frame}{Introduction}{Qu'est ce que le fuzzing ?}
\end{frame}


\begin{frame}{Plan}
  \tableofcontents
\end{frame}

\section{AFL}

\begin{frame}{AFL}{Algorithme génétique}
  \includegraphics[width=\textwidth]{../medias/schema_genetique.png}
\end{frame}

\begin{frame}{AFL}{Mutations}
  \begin{quote}\Large
  Designing the mutation engine for a new fuzzer has more to do with art than science
  \end{quote}
  \begin{exampleblock}{Stratégies déterministes}
    \begin{itemize}
      \item{bit flip : 1101 0010 devient 1100 0010} \pause
      \item{byte flip : 0xdeadbeef devient 0xde52beef} \pause
      \item{arithmétique simple : 0xdeadbeef devient 0xdeafbeef} \pause
      \item{constantes intéressantes : -1, 256, INT\_MAX, ...} \pause
    \end{itemize}
  \end{exampleblock}
  \begin{exampleblock}{Stratégies aléatoires : empilement de plusieurs petites modifications}
    \begin{itemize}
      \item{unique bit flip} \pause
      \item{remplacement d'un byte aléatoirement ou par un autre intéressant} \pause
      \item{suppression de bloc} \pause
      \item{duplication de bloc via suppression ou insertion} \pause
      \item{memset de bloc}
    \end{itemize}
  \end{exampleblock}
\end{frame}

\begin{frame}{AFL}{Basic Block}
  \begin{figure}
    \includegraphics[width=0.65\textwidth]{../medias/BB.png}
  \end{figure}
\end{frame}

\begin{frame}[fragile]{AFL}{Instrumentation}
  {\Large \centerline{L'instrumentation est un bout de code rajouté par AFL au programme}}
  \begin{itemize}
    \item{Elle est faite au niveau de chaque basic bloc} \pause
  \end{itemize}

  \hspace{0.5cm}

  \begin{lstlisting}
    cur_location = <COMPILE_TIME_RANDOM>;
    shared_mem[cur_location ^ prev_location]++;
    prev_location = cur_location >> 1;
  \end{lstlisting}

  \pause
  \hspace{3.5cm}

  \begin{exampleblock}{Explications}
    \begin{itemize}
      \item{Valeur aléatoire pour simplifier l'édition de lien} \pause
      \item{Une case du tableau shared\_mem[] correspond à une flèche du graphe précédent} \pause
      \item{Shift pour différencier A -> B de B -> A, ainsi que A -> A de B -> B}
    \end{itemize}
  \end{exampleblock}
\end{frame}

\begin{frame}{Nouveaux comportements}
  \begin{columns}
    \begin{column}{0.5\textwidth}
      \begin{exampleblock}{Nouvelles transitions}
        \begin{itemize}
          \item{Pas de comparaison des traces}
          \item{Comparaison des transitions}
          \item{Nouvelle transition intéressant}
          %\item{Shift pour différencier A -> B de B -> A, ainsi que A -> A de B -> B}
        \end{itemize}
      \end{exampleblock}
    \end{column}

    \begin{column}{0.5\textwidth}
      \begin{exampleblock}{Catégories}
        \begin{itemize}
          \item{shared\_mem[] rangé par catégories}
          \item{1, 2, 3, 4-7, 8-15, 16-31, 32-127, 128+} 
          \item{Changement de catégorie intéressant}
        \end{itemize}
      \end{exampleblock}
    \end{column}
  \end{columns}

  \pause
  \hspace{50.5cm}

  {\Large \centerline{Exemples}}

  \hspace{20.5cm}

  \begin{itemize}
    \item{\#1: A -> B -> B -> A -> C -> D} \pause
    \item{\#2: A -> B -> A -> B -> A -> C} \pause
    \item{\#3: A -> B -> C -> D} \pause
    \item{\#4: A -> B -> A -> B}
  \end{itemize}
\end{frame}

\begin{frame}{AFL}{Modèle "fork server" (1)}
  \begin{columns}[t]
    \begin{column}{0.49\textwidth}
      \begin{block}{"classique"}
        \begin{itemize}
        \item \lstinline{execve()}
          \begin{itemize}
          \item copie du programme en mémoire
          \item \lstinline{ld-linux.so} charge les librairies partagées
          \end{itemize}
        \item \lstinline{waitpid()}
        \item tout recommencer avec une autre entrée
        \end{itemize}
        \vspace{3.5ex}
      \end{block}
    \end{column}

    \begin{column}{0.49\textwidth}
      \begin{block}{"fork server"}
        \begin{itemize}
        \item \lstinline{execve()} "classique"
        \item programme cible instrumenté
          \begin{itemize}
          \item "pause" avant le \lstinline{main} du programme
          \item \lstinline{fork()} pour copier le processus
          \item \lstinline{waitpid()} pour surveiller les enfants
          \end{itemize}
        \item communication entre alf-fuzz et programme parent avec un \lstinline{pipe}
        \end{itemize}
      \end{block}
    \end{column}
  \end{columns}
\end{frame}

\begin{frame}{AFL}{Modèle "fork server" (2)}
  \begin{columns}[t]
    \begin{column}{0.5\textwidth}
      \begin{center}
        \textbf{"classique"}
      \end{center}

      \bigskip
      \begin{figure}
        \includegraphics[width=.4\textwidth]{../medias/classique.png}
      \end{figure}
    \end{column}

    \begin{column}{0.5\textwidth}
      \begin{center}
        \textbf{"fork server"}
      \end{center}

      \begin{figure}
        \includegraphics[width=.4\textwidth]{../medias/fork-server.png}
      \end{figure}
    \end{column}
  \end{columns}
\end{frame}

\section{Fuzzing de Radare2}

%%%%%%%%%%%%%%%%%%%%%%%%%%%%%%%%%%%%%%%%%%%%%%%%%%%%%%%%%%%%%%%%%%%%%%%%%
% Présentation de Radare2
%%%%%%%%%%%%%%%%%%%%%%%%%%%%%%%%%%%%%%%%%%%%%%%%%%%%%%%%%%%%%%%%%%%%%%%%%
\begin{frame}{Fuzzing de Radare2}{Pourquoi Radare2 ?}
  \includegraphics[width=\linewidth]{../medias/radare2-github.png}
  \begin{exampleblock}{Radare2 en quelques mots}
    \begin{itemize}
    \item{outil de reverse-engineering opensource (GPLv3)}
    \item{40+ architectures (x86, ARM, MIPS, SPARC...)}
    \item{30+ formats (ELF, PE, DEX...)}
    \item{écrit en C (bugs de corruption mémoire)}
    \item{20000+ commits, 600+ contributeurs, 900K+ LoC}
    \end{itemize}
  \end{exampleblock}
\end{frame}

%%%%%%%%%%%%%%%%%%%%%%%%%%%%%%%%%%%%%%%%%%%%%%%%%%%%%%%%%%%%%%%%%%%%%%%%%
% Présentation de la méthodologie
%%%%%%%%%%%%%%%%%%%%%%%%%%%%%%%%%%%%%%%%%%%%%%%%%%%%%%%%%%%%%%%%%%%%%%%%%
\begin{frame}[fragile]{Fuzzing de Radare2}{Méthodologie}
  \begin{block}{Méthodologie}
    \begin{itemize}
    \item{installation de AFL}
    \item{instrumentation de Radare2 grâce à \lstinline{afl-gcc}}
    \item{établir une base d'entrées initiales pour alimenter le fuzzer}
    \item{analyse des résultats avec Address Sanitizer}
    \end{itemize}
  \end{block}

  \pause

  \begin{exampleblock}{Récupération des sources}
    \begin{itemize}
    \item{\url{http://lcamtuf.coredump.cx/afl/releases/afl-latest.tgz}}
    \item{\url{https://github.com/radare/radare2}}
    \end{itemize}
  \end{exampleblock}

  \pause
  \vfill

  \begin{exampleblock}{Instrumentation de radare2}
    \begin{itemize}
    \item \lstinline{CC=afl-gcc ./sys/user.sh --install-path ../radare2-install}
    \end{itemize}
  \end{exampleblock}

\end{frame}


%%%%%%%%%%%%%%%%%%%%%%%%%%%%%%%%%%%%%%%%%%%%%%%%%%%%%%%%%%%%%%%%%%%%%%%%%
% Entrées de Radare2
%%%%%%%%%%%%%%%%%%%%%%%%%%%%%%%%%%%%%%%%%%%%%%%%%%%%%%%%%%%%%%%%%%%%%%%%%
\begin{frame}[fragile]{Fuzzing de Radare2}{Génération des entrées}
  \begin{center}
    \includegraphics[width=0.25\textwidth, clip=true]{../medias/radare2-logo.png}
  \end{center}

  \begin{columns}[T]
    \begin{column}{0.44\linewidth}

      \begin{block}{Analyseur de fichiers}
        \begin{itemize}
        \item{\path{binary-samples/}}
        \item{\path{radare2-regressions/}}
        \item{sélection de petits binaires}
        \item{$\ne$ architectures/formats}
        \end{itemize}

        \vspace{1.75ex}
      \end{block}
    \end{column}

    \begin{column}{0.44\linewidth}
      \begin{block}{Interpréteur de commandes}
        \begin{itemize}
        \item{1 caractère = 1 action}
        \item{\lstinline{px} = ``Print hexdump''}
        \item{\lstinline{pxj} = ``Print hexdump in JSON''}
        \item{élaboration de scripts}
        \end{itemize}
      \end{block}
    \end{column}
  \end{columns}
  \pause
  \vfill
  \begin{exampleblock}{Autres outils}
    \begin{itemize}
    \item{\lstinline{rax2}, \lstinline{rabin2}, \lstinline{rafind2}, \lstinline{rasm2}, \lstinline{radiff2}, ...}
    \end{itemize}
  \end{exampleblock}
\end{frame}

%%%%%%%%%%%%%%%%%%%%%%%%%%%%%%%%%%%%%%%%%%%%%%%%%%%%%%%%%%%%%%%%%%%%%%%%%
% afl-fuzz en pratique
%%%%%%%%%%%%%%%%%%%%%%%%%%%%%%%%%%%%%%%%%%%%%%%%%%%%%%%%%%%%%%%%%%%%%%%%%
\begin{frame}{Fuzzing de Radare2}{afl-fuzz}
  \begin{figure}
    \includegraphics[width=0.98\linewidth]{../medias/afl-overview.png}
  \end{figure}

  \pause

  \begin{exampleblock}{afl-fuzz}
    \begin{itemize}
    \item{serveurs de calculs du CREMI : 48 CPU + 128 Go de RAM}
    \item{10 instances d'\lstinline{afl-fuzz}}
    \item{en tout, environ 1 semaine de fuzzing}
    \end{itemize}
  \end{exampleblock}

\end{frame}

\begin{frame}{Fuzzing de Radare2}{afl-fuzz}
  \includegraphics[width=\linewidth]{../medias/afl-fuzz.png}
\end{frame}

%%%%%%%%%%%%%%%%%%%%%%%%%%%%%%%%%%%%%%%%%%%%%%%%%%%%%%%%%%%%%%%%%%%%%%%%%
% Résultats
%%%%%%%%%%%%%%%%%%%%%%%%%%%%%%%%%%%%%%%%%%%%%%%%%%%%%%%%%%%%%%%%%%%%%%%%%
\begin{frame}{Fuzzing de Radare2}{Résultats}
  \begin{exampleblock}{Address Sanitizer (ASan)}
    \begin{itemize}
    \item{développé par Google}
    \item{instrumentation basée sur LLVM}
    \item{détecte certaines corruptions mémoires (overflows, UAF...)}
    \end{itemize}
  \end{exampleblock}

  \pause

  \begin{block}{Résultats}
    \begin{itemize}
    \item{10 bugs remontés et corrigés par Radare2}
    \item{3 read out-of-bounds}
    \item{2 double-free}
    \item{2 bad-pointer-dereference}
    \item{1 use-after-free}
    \item{1 stack-based-overflow}
    \item{1 integer-overflow}
    \end{itemize}
  \end{block}
\end{frame}

\section{Binaires quelconques et ciblant d'autres plateformes}

\begin{frame}{Binaires quelconques et ciblant d'autres plateformes}{Problème}
  \lstinline{afl-gcc} pour instrumenter le binaire à la compilation

  \medskip
  \begin{itemize}
  \item Comment fuzzer des programmes sans avoir accès à leur code source ?
  \item Comment fuzzer des binaires ciblant d'autres plateformes ?
  \end{itemize}

  \medskip
  \begin{center}
    {\large$\longrightarrow$ QEMU mode}
  \end{center}
\end{frame}

\begin{frame}{Binaires quelconques et ciblant d'autres plateformes}{QEMU}
  \begin{quote}
  ``QEMU est un logiciel libre de machine virtuelle, pouvant émuler un processeur et, plus généralement, une architecture différente si besoin.''
  \end{quote}

  \medskip
  $\longrightarrow$ Exécuter un binaire pour une architecture \textbf{cible} sur notre architecture \textbf{hôte} (x86).
\end{frame}

\begin{frame}{Binaires quelconques et ciblant d'autres plateformes}{émulation en espace utilisateur}
  \begin{itemize}
  \item émule uniquement le CPU
  \item exécute le binaire
  \end{itemize}

  \begin{figure}
    \includegraphics[width=.9\textwidth,height=.4\textheight]{../medias/qemu.png}
  \end{figure}

  \begin{itemize}
  \item cache les blocs traduits (Translated Blocs)
  \item intercepte les appels systèmes pour les transmettre au noyau de l'hôte
  \end{itemize}
\end{frame}

\begin{frame}[fragile]{Binaires quelconques et ciblant d'autres plateformes}{AFL avec le mode QEMU}
  \begin{itemize}
  \item version patchée de QEMU
  \item instrumentation "dynamique" des blocs traduits
  \end{itemize}

  \begin{lstlisting}[language=C]
    if (block_address > elf_text_start
        && block_address < elf_text_end) {
      cur_location = (block_address >> 4) ^ (block_address << 8);
      shared_mem[cur_location ^ prev_location]++;
      prev_location = cur_location >> 1;
    }
  \end{lstlisting}

  \begin{itemize}
  \item "fork server" et cache des blocs traduits partagé
  \item deux à cinq fois plus lent
  \end{itemize}
\end{frame}

\begin{frame}{Binaires quelconques et ciblant d'autres plateformes}{Autres versions d'AFL}
  \begin{block}{android-afl (*nix system)}
    \begin{itemize}
    \item adapter \lstinline{afl-gcc} pour cibler l'architecture ARM
    \item adapter \lstinline{afl-as} pour injecter de l'ARMv7
    \item remplacer \lstinline{sys/shm.h} (non disponible sous Android)
    \end{itemize}
  \end{block}

  \begin{exampleblock}{WinAFL (ProjectZero)}
    \begin{itemize}
    \item \sout{instrumentation}, \sout{fork server}
    \item instrumentation dynamique avec \lstinline{DynamoRIO}
    \end{itemize}
  \end{exampleblock}
\end{frame}


\end{document}
