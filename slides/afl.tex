\section{AFL}

\begin{frame}{AFL}{Présentation}
\end{frame}

\begin{frame}{AFL}{Algorithme génétique}
\end{frame}

\begin{frame}{AFL}{Instrumentation}
\end{frame}

\begin{frame}{AFL}{Modèle "fork server" (1)}
  \begin{columns}[t]
    \begin{column}{0.5\textwidth}
      \begin{center}
        \textbf{"classique"}
      \end{center}

      \begin{itemize}
      \item \lstinline{execve()}
        \begin{itemize}
        \item copie du programme en mémoire
        \item \lstinline{ld-linux.so} charge les librairies partagées
        \end{itemize}
      \item \lstinline{waitpid()}
        \begin{itemize}
        \item attendre que le programme termine ou crash
        \end{itemize}
      \item tout recommencer avec une autre entrée
      \end{itemize}
    \end{column}

    \begin{column}{0.5\textwidth}
      \begin{center}
        \textbf{"fork server"}
      \end{center}

      \begin{itemize}
      \item \lstinline{execve()} "classique"
      \item programme cible instrumenté
        \begin{itemize}
        \item "pause" avant le \lstinline{main} du programme
        \item \lstinline{fork()} pour copier le processus
        \item \lstinline{waitpid()} pour surveiller les enfants
        \end{itemize}
      \item communication entre alf-fuzz et programme parent avec un \lstinline{pipe}
      \end{itemize}
    \end{column}
  \end{columns}
\end{frame}

\begin{frame}{AFL}{Modèle "fork server" (2)}
  \begin{columns}[t]
    \begin{column}{0.5\textwidth}
      \begin{center}
        \textbf{"classique"}
      \end{center}

      \bigskip
      \begin{figure}
        \includegraphics[width=.4\textwidth]{../medias/classique.png}
      \end{figure}
    \end{column}

    \begin{column}{0.5\textwidth}
      \begin{center}
        \textbf{"fork server"}
      \end{center}

      \begin{figure}
        \includegraphics[width=.4\textwidth]{../medias/fork-server.png}
      \end{figure}
    \end{column}
  \end{columns}
\end{frame}
