\chapter{Ouverture}

Notre étude explique le fonctionnement d'AFL, d'ASan, et nous présentons une
application concrète sur une suite de logiciels moderne : Radare2.

La rapidité et la facilité avec laquelle AFL expose des problèmes confirme
l'efficacité du fuzzing comme outil pour trouver des bugs dans un programme.
L'utilisation complémentaire d'outils comme ASan apporte des résultats
supplémentaires et se combine très bien avec l'utilisation d'un fuzzer.

AFL n'est cependant pas le seul outil utilisé dans le cadre du fuzzing.
Développé par Google à l'origine pour son navigateur Chrome, ClusterFuzz
est une infrastructure de fuzzing destinée à accompagner la recherche de
bugs en continue lors des phases de développement d'un programme.
Lancé en 2016, OSS-Fuzz\footnote{\url{https://github.com/google/oss-fuzz}}
est une plateforme utilisant ClusterFuzz comme back end, mise en place pour
aider les projets open source à trouver des bugs. Le code source de
ClusterFuzz\footnote{\url{https://github.com/google/clusterfuzz}}
a d'ailleurs été lui-même ouvert début février 2019.

Le moteur de fuzzing libFuzzer\footnote{\url{https://llvm.org/docs/LibFuzzer.html}}
utilise quant à lui l'infrastructure de compilation LLVM, et en particulier
l'instrumentation apportée par son outil SanitizerCoverage, pour son
exploration du code à tester. Ce fuzzer nécessite de créer un point d'entrée
dans le programme à tester, une "fonction cible", à laquelle libFuzzer va
alors s'attaquer pour chercher des bugs.

Frustré par le nombre de bugs présents dans les projets open source, Hanno
Böck a de son côté aussi souhaité aider la communauté en lançant
le Fuzzing Project\footnote{\url{https://fuzzing-project.org}}. L'idée étant
d'appliquer lui même une série de tests sur une partie des projets open
source les plus utilisés et de remonter les bugs aux contributeurs concernés.
