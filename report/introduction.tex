\chapter*{Introduction}
\addcontentsline{toc}{chapter}{\protect\numberline{}Introduction}

Ce projet de recherche vise à étudier des techniques de fuzzing logiciel. En
particulier, on s'intéresse à American Fuzzy Lop (AFL), un des programmes les
plus utilisés et performants dans le domaine.

Nous présentons en première partie les principes généraux du fuzzing, ainsi que
les rouages internes d'AFL. L'accent est mis sur le fonctionnement de son
algorithme génétique, ainsi que l'instrumentation qu'il met en place à la
compilation d'un programme pour guider sa recherche. Nous expliquons également
les méthodes astucieuses qu'il utilise pour optimiser sa vitesse d'exécution
et ses performances, comme le fork server.

En deuxième partie, nous utilisons AFL pour fuzzer une suite d'outils :
Radare2. L'idée est de voir en pratique comment l'installer, l'utiliser
pour attaquer un programme, et en analyser les résultats.
L'utilisation d'un logiciel supplémentaire, ASan, permet d'obtenir une
analyse encore plus fine des résultats et même de trouver des bugs qu'AFL ne
voit pas.

La troisième partie va un peu plus loin dans les possibilités d'AFL, et
explique comment, même sans le code source du binaire à tester, on peut
instrumenter ce dernier pour pouvoir le fuzzer. Pour cela, AFL utilise l'outil
de virtualisation QEMU et en particulier son émulation en espace utilisateur.
