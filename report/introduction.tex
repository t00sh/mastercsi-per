\chapter*{Introduction}

Ce projet de recherche vise à étudier des techniques de fuzzing logiciel. En
particulier, on s'intéresse au logiciel AFL, un des plus utilisé dans le
domaine.

Nous présentons en première partie le fuzzing, ainsi que les rouages internes
d'AFL. L'accent est mis sur le fonctionnement de son algorithme génétique,
ainsi que l'instrumentation qu'il met en place à la compilation
d'un binaire pour guider sa recherche. Nous expliquons également les
méthodes astucieuses qu'il utilise pour optimiser sa vitesse d'exécution
et ses performances, comme le fork server.

En deuxième partie, nous utilisons AFL pour fuzzer une suite d'outils :
Radare2. L'idée est de voir en pratique comment installer, utiliser
le fuzzer sur un binaire, et analyser ses résultats.

Nous analysons les résultats et bugs trouvés grâce à AFL. L'utilisation
d'un logiciel supplémentaire, ASan, permet d'obtenir une analyse encore
plus fine des résultats et même de trouver des bugs qu'AFL ne voit pas.

La troisième partie va un peu plus loin dans les possibilités d'AFL, et
explique comment, même sans le code source du binaire à tester, on peut
instrumenter ce dernier pour pouvoir le fuzzer. AFL se servant pour cela de
l'outil de virtualisation QEMU et en particulier de son émulation en
espace utilisateur.

