\documentclass{bredelebeamer}

\title[]{Techniques de fuzzing}
\subtitle{Avec American Fuzzy Lop}

\author[Xavier M. - Brendan G. - Simon D.]{Xavier Maso \\ Brendan Guevel \\ Simon Duret}


\institute[]{
  \includegraphics[scale=0.2]{../medias/universite-bordeaux.pdf}
}

\date{XX février 2019}

\begin{document}

\begin{frame}
  \titlepage
\end{frame}

%-------------------------------------------------------------------------------

\begin{frame}{Introduction}{Qu'est ce que le fuzzing ?}
\end{frame}

%-------------------------------------------------------------------------------

\begin{frame}{Plan}
  \tableofcontents
\end{frame}

%-------------------------------------------------------------------------------

\section{AFL}

\begin{frame}{AFL}{Présentation}
\end{frame}

\begin{frame}{AFL}{Algorithme génétique}
\end{frame}

\begin{frame}{AFL}{Instrumentation}
\end{frame}

\begin{frame}{AFL}{Modèle "fork server"}
\end{frame}

%-------------------------------------------------------------------------------

\section{Fuzzing de Radare2}

\begin{frame}{Fuzzing de Radare2}{Mise en place}
\end{frame}

\begin{frame}{Fuzzing de Radare2}{Résultats}
\end{frame}

%-------------------------------------------------------------------------------

\section{Binaires quelconques et ciblant d'autres plateformes}

\begin{frame}{Binaires quelconques et ciblant d'autres plateformes}{Problème}
\end{frame}

\begin{frame}{Binaires quelconques et ciblant d'autres plateformes}{QEMU}
\end{frame}

\begin{frame}{Binaires quelconques et ciblant d'autres plateformes}{AFL avec le mode QEMU}
\end{frame}

\begin{frame}{Binaires quelconques et ciblant d'autres plateformes}{Autres versions d'AFL}
\end{frame}

%-------------------------------------------------------------------------------

\section*{Conclusion}

\begin{frame}{Conclusion}
\end{frame}

\end{document}
